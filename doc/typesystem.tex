\documentclass{article}

\usepackage{amsmath}
\usepackage{amssymb}
\usepackage{mathtools}

\usepackage{listings}
\usepackage{proof}
\usepackage{framed}

\newcommand{\ldot}{\mathpunct{.}}
\newcommand{\nats}{\mathbb{N}}
\newcommand{\type}[1]{\texttt{#1}}

\title{Type System}
\author{Leander Tentrup}

\begin{document}

\maketitle

\section{Preliminaries}

\begin{itemize}
  \item A stream has two types, a \emph{value type} representing the value stored, e.g., \type{Bool}, \type{String}, and \type{Int32}, and a \emph{stream type} representing when a stream is scheduled.
  \item A stream type can be either \emph{periodic} with a fixed frequency, or input-driven with a variable frequency.
  \item For a stream with variable frequency, we define an activation condition, that is, a Boolean constraint, and a stream is evaluated only if the activation condition evaluates to true.
  \item A type rule describes how the stream and value type for an expression is derived.
\end{itemize}

\begin{framed}
Example specification
\begin{lstlisting}
input a: Bool, b: Bool, c: Bool;
output x = a || b;
output y = x && c;
\end{lstlisting}
\end{framed}

\begin{framed}
Example type rule
\begin{equation*}
  \infer
  {\Gamma \vdash e_0 + e_1 \colon \binom{\tau'}{\sigma'}}
  {\Gamma \vdash e_0 \colon \binom{\tau_0}{\sigma_0} & \Gamma \vdash e_1 \colon \binom{\tau_1}{\sigma_1} & \tau' \sqsubseteq \tau_0 \sqcap \tau_1 & \tau' \sqsubseteq \type{Numeric} & \sigma' \sqsubseteq \sigma_0 \sqcap \sigma_1}
\end{equation*}
Here, $\tau$ is a value type, $\sigma$ is a stream type, and $\binom{\tau}{\sigma}$ is the combined type.
\end{framed}

\subsection{Value Types}

For value types the following holds
\begin{itemize}
  \item $\type{Int8} \sqsubset \type{Int16} \sqsubset \type{Int32} \sqsubset \type{Int64}$,
  \item $\type{UInt8} \sqsubset \type{UInt16} \sqsubset \type{UInt32} \sqsubset \type{UInt64}$, and
  \item $\type{Float32} \sqsubset \type{Float64}$
\end{itemize}

Type classes
\begin{itemize}
  \item \type{UnsignedInteger}: \type{UIntX}
  \item \type{SignedInteger}: \type{IntX}
  \item $\type{Integer} = \type{UnsignedInteger} \cup \type{SignedInteger}$
  \item \type{FloatingPoint}: \type{FloatX}
  \item $\type{Numeric} = \type{Integer} \cup \type{FloatingPoint}$
  \item \type{Comparable}, \type{Equatable}: Currently all types
\end{itemize}

\subsection{Stream Types}

For two periodic $\sigma_0$ and $\sigma_1$ with frequencies $f_0$ and $f_1$, respectively, it holds that
\begin{equation*}
  \sigma_0 \sqcap \sigma_1 \coloneqq
  \begin{cases}
    f_0 & \text{if } f_0 \sqsubseteq f_1 \\
    f_1 & \text{if } f_1 \sqsubseteq f_0 \\
    \bot & \text{otherwise}
  \end{cases}
\end{equation*}
where $f_0 \sqsubseteq f_1$ is defined as $\exists c \in \nats \colon f_0 \cdot c = f_1$.
%
For example, $1\,\text{Hz} \sqcap 2\,\text{Hz} = 1\,\text{Hz}$, $4\,\text{Hz} \sqcap 2\,\text{Hz} = 2\,\text{Hz}$, and $2\,\text{Hz} \sqcap 3\,\text{Hz} = \bot$.

For two variable frequency streams, $\sqcap$ amounts to a Boolean conjunction of the activation conditions of the operands.
%
For example, \texttt{a || b} returns $\sigma \sqsubseteq a \land b$. and \texttt{x \&\& c} returns $\sigma \sqsubseteq x \land c = a \land b \land c$.
$\sigma \sqsubseteq \sigma'$ holds if $\sigma \rightarrow \sigma'$ is valid.
Disjunctive constraints can be produce by annotating the output declaration, e.g.,
\begin{lstlisting}
output z @(x | y) := x!false xor y!false;
\end{lstlisting}
produces an output stream $z$ that is evaluates if either $x$ or $y$ are evaluated.
As $(x \lor y) \rightarrow x$ is not valid, we have to use the sample and hold operator $!$ as it is not guaranteed that $x$ and $y$ have values whenever $z$ is evaluated.

\end{document}
